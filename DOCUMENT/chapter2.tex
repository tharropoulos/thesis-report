\chapter{Συμπεράσματα}
\label{ch:chapter5}

Η παρούσα διπλωματική εργασία μελέτησε την αξιολόγηση της ποιότητας του κώδικα που παράγεται από Μεγάλα Γλωσσικά Μοντέλα, εστιάζοντας στο \tl{GitHub Copilot}, και την βελτιστοποίηση των προτροπών για την επίτευξη των επιθυμητών αποτελεσμάτων. Τα βασικά συμπεράσματα της έρευνας συνοψίζονται στα εξής:

\section{Αξιολόγηση Ποιότητας Κώδικα}
\subsection*{Απόδοση ανά Τύπο Κώδικα}
\begin{itemize}
    \item Στον έλεγχο μονάδας \tl{(Unit Testing)}, το μοντέλο παρουσίασε υψηλή απόδοση
    \item Στον έλεγχο ενσωμάτωσης \tl{(Integration Testing)}, παρατηρήθηκε πιο ισορροπημένη κατανομή μεταξύ μικρών και μεγάλων αλλαγών
    \item Στην ανάπτυξη του \tl{Backend}, οι επεμβάσεις ήταν λιγότερες αλλά κυρίως σε μεγάλη έκταση κώδικα
\end{itemize}

\subsection*{Περιορισμοί και Προκλήσεις}
\begin{itemize}
    \item Το μοντέλο δυσκολεύτηκε ιδιαίτερα με πολύπλοκες σχέσεις στη βάση δεδομένων (Μ:Ν)
    \item Παρατηρήθηκαν δυσκολίες στην κατανόηση και υλοποίηση περίπλοκων επιχειρησιακών λογικών
    \item Η διαχείριση των ορίων συμβόλων \tl{(token limits)} απαιτούσε ειδική μεταχείριση
\end{itemize}

\section{Βελτιστοποίηση Προτροπών}

Η ανάλυση των προτροπών και των αποτελεσμάτων τους οδήγησε στα εξής συμπεράσματα:

\subsection*{Απόδοση Απαντήσεων}
\begin{itemize}
    \item 55.34\% των απαντήσεων αξιολογήθηκαν θετικά
    \item 43.51\% των απαντήσεων αξιολογήθηκαν αρνητικά
    \item Η συνέχεια των θετικών αξιολογήσεων ήταν υψηλότερη από των αρνητικών
\end{itemize}

\subsection*{Μοντέλο Πρόβλεψης}
\begin{itemize}
    \item Επιτεύχθηκε ακρίβεια 70.10\% στην πρόβλεψη της ποιότητας των απαντήσεων
    \item Οι αλγόριθμοι \tl{Random Forest} και \tl{Naive Bayes} παρουσίασαν την καλύτερη απόδοση
    \item Η προσέγγιση νευρωνικών δικτύων δεν απέδωσε λόγω περιορισμένου όγκου δεδομένων
\end{itemize}

\section{Προτάσεις για Μελλοντική Έρευνα}

Με βάση τα ευρήματα της έρευνας, προτείνονται οι εξής κατευθύνσεις για μελλοντική έρευνα:

\subsection*{Βελτίωση Συλλογής Δεδομένων}
\begin{itemize}
    \item Ανάπτυξη μεθοδολογίας για τη συλλογή μεγαλύτερου όγκου δεδομένων αξιολόγησης
    \item Διερεύνηση μεθόδων για την αυτοματοποιημένη αξιολόγηση της ποιότητας του κώδικα
\end{itemize}

\subsection*{Εξέλιξη Μοντέλου Πρόβλεψης}
\begin{itemize}
    \item Διερεύνηση πιο προηγμένων τεχνικών προεπεξεργασίας δεδομένων
    \item Πειραματισμός με συνδυαστικές μεθόδους μηχανικής μάθησης
    \item Ενσωμάτωση περισσότερων χαρακτηριστικών στην ανάλυση των προτροπών
\end{itemize}

\subsection*{Ηθικά Ζητήματα και Προκλήσεις}
\begin{itemize}
    \item Διερεύνηση των ηθικών ζητημάτων στη συλλογή και χρήση δεδομένων
    \item Μελέτη της επίδρασης των αδειών χρήσης στην ανάπτυξη και χρήση των μοντέλων
    \item Ανάπτυξη πλαισίου για την προστασία της πνευματικής ιδιοκτησίας
\end{itemize}

\section{Τελικά Συμπεράσματα}

Η έρευνα κατέδειξε ότι τα Μεγάλα Γλωσσικά Μοντέλα, συγκεκριμένα το \tl{GitHub Copilot}, αποτελούν ισχυρά εργαλεία υποστήριξης της ανάπτυξης λογισμικού, αλλά απαιτούν προσεκτική καθοδήγηση και κατανόηση των περιορισμών τους. Η βελτιστοποίηση των προτροπών αποτελεί κρίσιμο παράγοντα για την αποτελεσματική χρήση τους, ενώ η ανάπτυξη μοντέλων πρόβλεψης της απόδοσής τους προσφέρει σημαντικές δυνατότητες για τη βελτίωση της αποτελεσματικότητάς τους.

Η συνεχής εξέλιξη του πεδίου της Τεχνητής Νοημοσύνης και των Μεγάλων Γλωσσικών Μοντέλων δημιουργεί νέες προκλήσεις και ευκαιρίες για έρευνα και ανάπτυξη. Η κατανόηση και αντιμετώπιση των ηθικών ζητημάτων που προκύπτουν θα είναι καθοριστική για τη μελλοντική ανάπτυξη και χρήση αυτών των τεχνολογιών.