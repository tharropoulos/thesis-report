\chapter{Εισαγωγή}
\label{ch:chapter1}




Η ανάπτυξη των Μεγάλων Γλωσσικών Μοντέλων \textlatin{(LLM)} έχει επιφέρει ριζικές αλλαγές στον τομέα της Τεχνητής Νοημοσύνης και της Επεξεργασίας Φυσικής Γλώσσας \textlatin{(NLP)} \cite{bommasani2021opportunities,zhao2023survey,zhou2023comprehensive}. Η εισαγωγή αυτών των μοντέλων στην καθημερινή ζωή μέσω του \textlatin{Chat-GPT} το 2022 \cite{openai2022chatgpt} έχει πυροδοτήσει μια επανάσταση στην τεχνολογική αγορά, σηματοδοτώντας την απαρχή του αγώνα για την κυριαρχία στην αγορά της Τεχνητής Νοημοσύνης \cite{guardian2024openai,nyt2024openai,verge2023chatgpt,liu2023chatgpt,trendforce2023ai}. Τα μοντέλα αυτά έχουν αποδειχθεί εξαιρετικά αποτελεσματικά στην αντιμετώπιση προβλημάτων επεξεργασίας φυσικής γλώσσας, όπως η αναγνώριση φυσικής γλώσσας, προάγοντας την ανάπτυξη της Γενικής Τεχνητής Νοημοσύνης \textlatin{(AGI)} \cite{adams2012mapping,goertzel2014agi}.

Μέσα σε αυτή την επανάσταση, η χρήση μοντέλων επεξεργασίας φυσικής γλώσσας στον τομέα του προγραμματισμού και της ανάπτυξης λογισμικού έχει αναδειχθεί ως ένας από τους πιο υποσχόμενους τομείς της τεχνολογίας. Ένα από τα πιο διαδεδομένα εργαλεία με αυτόν το σκοπό είναι το \textlatin{GitHub Copilot} \cite{github2021copilot}. Αναπτυγμένο σε συνεργασία με την \textlatin{OpenAI}, το \textlatin{GitHub Copilot} ξεκίνησε χρησιμοποιώντας το μοντέλο ονόματι \textlatin{Codex} της \textlatin{OpenAI}\cite{chen2021evaluating}, σχεδιασμένο εξ' αρχής αποκλειστικά για τη παραγωγή κώδικα, για να προτείνει κώδικα στον προγραμματιστή κατά την γραφή κώδικα.

Η παροδική απόσυρση του μοντέλου \textlatin{Codex} τον Μάρτιο του 2023 και η οριστική του απόσυρση το 2023 \cite{kemper2023openai}, οδήγησε στην ανάπτυξη ενός νέου μοντέλου, σε συνεργασία μεταξύ της \textlatin{OpenAI}, της \textlatin{Microsoft Azure AI}, και της \textlatin{GitHub AI}. Το νέο μοντέλο βασίζεται στο \textlatin{GPT-3.5 Turbo}\cite{github2023copilotupdate}
\subsection{Ακόμα ένας τίτλος}