\begin{titlepage}

\begin{figure}[H]
  \begin{center}
    \includegraphics[width=3cm]{auth.pdf}
    \label{fig:cover_auth_logo}
  \end{center}
\end{figure}

\centering
\Large Αριστοτέλειο Πανεπιστήμιο Θεσσαλονίκης\\
\Large Πολυτεχνική Σχολή\\
\large Τμήμα Ηλεκτρολόγων Μηχανικών και Μηχανικών Υπολογιστών\\
\large Τομέας Τηλεπικοινωνιών

\vspace{\fill}

\LARGE Τίτλος διπλωματικής

\vspace{\fill}

\Large Διπλωματική Εργασία\\
\Large του\\
\Large Θεοφάνη Θαρρόπουλου

\vspace{\fill}
\raggedright

\begin{tabular}{ll}
\textbf{Επιβλέπων:} & Ανδρέας Συμεωνίδης \\
 & Καθηγητής Α.Π.Θ.\\
\end{tabular}

\centering
\vspace{\fill}
\today


\end{titlepage}

\begin{abstract}
Αντικείμενο της παρούσας διπλωματικής εργασίας αποτελεί η έρευνα για την αξιολόγηση της ποιότητας του κώδικα που παράγεται από Μεγάλα Γλωσσικά Μοντέλα \tl{(LLMs)}, και πιο συγκεκριμένα από το \textlatin{GitHub Copilot}\cite{githubcopilot}. Η μελέτη εστιάζει στην αξιολόγηση της ποιότητας του κώδικα που παράγεται από το Copilot και στην βελτιστοποίηση των προτροπών \textlatin{(prompts)} για την επίτευξη των επιθυμητών αποτελεσμάτων μέσω τεχνικών μηχανικής προτροπής \textlatin{(prompt engineering)} και της μηχανικής μάθησης. Τα αποτελέσματα αποδεικνύουν τις δυνατότητες και τους περιορισμούς του Copilot στην παραγωγή ποιοτικού κώδικα και προσφέρουν νέες προσεγγίσεις για την βελτίωση της αλληλεπίδρασης μεταξύ του χρήστη και του εργαλείου μέσω στοχευμένων τεχνικών προτροπής.
\end{abstract}

\selectlanguage{english}
\begin{abstract}
Empty
\end{abstract}

\thispagestyle{empty}

\selectlanguage{greek}

\section*{Ευχαριστίες}
\thispagestyle{empty}

Άδειο

\clearpage
