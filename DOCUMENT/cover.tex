\begin{titlepage}

  \begin{figure}[H]
    \begin{center}
      \includegraphics[width=3cm]{auth.pdf}
      \label{fig:cover_auth_logo}
    \end{center}
  \end{figure}

  \centering \Large Αριστοτέλειο Πανεπιστήμιο Θεσσαλονίκης\\ \Large
  Πολυτεχνική Σχολή\\ \large Τμήμα Ηλεκτρολόγων Μηχανικών και Μηχανικών
  Υπολογιστών\\ \large Τομέας Ηλεκτρονικής και Υπολογιστών \\ \large Ομάδα Ευφυών Συστημάτων και Τεχνολογίας Λογισμικού \tl{(ISSEL)}

  \vspace{
    \fill}

  \LARGE Αξιολόγηση ποιότητας κώδικα παραγόμενου από Μεγάλα Γλωσσικά Μοντέλα και βελτιστοποίηση προτροπών
 

  \vspace{
    \fill}

  \Large Διπλωματική Εργασία\\ \Large του\\ \Large Θεοφάνη Θαρρόπουλου\\ \Large ΑΕΜ 9914

  \vspace{
    \fill} \raggedright

  \begin{tabular}{ll}
    \textbf{Επιβλέπων:} & Ανδρέας Συμεωνίδης \\
    & Καθηγητής Α.Π.Θ.\\
  \end{tabular}

  \centering \vspace{
    \fill} \today

\end{titlepage}

\begin{abstract}
  Αντικείμενο της παρούσας διπλωματικής εργασίας αποτελεί η έρευνα για
  την αξιολόγηση της ποιότητας του κώδικα που παράγεται από Μεγάλα
  Γλωσσικά Μοντέλα \tl{(LLMs)}, και πιο συγκεκριμένα από το
  \textlatin{GitHub Copilot} \cite{githubcopilot}. Η μελέτη εστιάζει στην
  αξιολόγηση της ποιότητας του κώδικα που παράγεται από το \tl{Copilot} και
  στην βελτιστοποίηση των προτροπών \textlatin{(prompts)} για την
  επίτευξη των επιθυμητών αποτελεσμάτων μέσω τεχνικών μηχανικής
  προτροπής \textlatin{(prompt engineering)} της μηχανικής μάθησης.
  Τα αποτελέσματα αποδεικνύουν τις δυνατότητες και τους περιορισμούς του
  \tl{Copilot} στην παραγωγή ποιοτικού κώδικα και προσφέρουν νέες
  προσεγγίσεις για την βελτίωση της αλληλεπίδρασης μεταξύ του χρήστη και
  του εργαλείου μέσω στοχευμένων τεχνικών προτροπής.
\end{abstract}

\thispagestyle{empty}

\selectlanguage{english}
\clearpage
\begin{center}
{\textbf{Title}} \\
\end{center}
\begin{center}
Upon evaluating source code generated by LLMs and improving the prompt engineering process \\[1cm]
\end{center}
\begin{center}
 \textbf{Abstract}
\end{center}

The objective of this thesis is to investigate the quality assessment of code generated by Large Language Models (LLMs), specifically GitHub Copilot \cite{githubcopilot}. The study focuses on evaluating the quality of code produced by Copilot and optimizing prompts to achieve desired outcomes through machine learning prompt engineering techniques. The results demonstrate Copilot's capabilities and limitations in producing quality code and offer new approaches for improving user-tool interaction through targeted prompting techniques.

\clearpage

\selectlanguage{greek}

\section*{Ευχαριστίες} \thispagestyle{empty}
Θα ήθελα να ευχαριστήσω τον καθηγητή κ. Ανδρέα Συμεωνίδη για την ευκαιρία να εργαστώ με αυτόν πάνω σε ένα θέμα που με ενδιαφέρει τόσο πολύ. Θα ήθελα επίσης να τον ευχαριστήσω για την βοήθεια που έλαβα καθ' όλη την διάρκεια της διπλωματικής μου διατριβής, την κατανόηση και την υποστήριξη που μου προσέφερε.
Θα ήθελα επίσης να εκφράσω τις ευχαριστίες μου προς τους φίλους μου, οι οποίοι με υποστήριξαν και με βοήθησαν κατά την διάρκεια της εκπόνησης της διπλωματικής μου εργασίας.
Θα ήθελα να ευχαριστήσω τους γονείς μου, για όλες τις θυσίες που έκαναν προκειμένου να μπορέσω να σπουδάσω αυτό που αγαπώ.
Τέλος θα ήθελα να ευχαριστήσω τον αδερφό μου, που με ενέπνευσε να ακολουθήσω το επάγγελμα του μηχανικού, που αποτέλεσε παράδειγμα προς μίμηση καθ' όλη την διάρκεια της ζωής μου και με βόηθησε και συνεχίζει συστηματικά να με βοηθάει στην πορεία μου ως μηχανικός λογισμικού. Η αγάπη που έχω για το πρόσωπό του είναι πολύ μεγαλύτερη από αυτό που μπορώ να εκφράσω, και χωρίς αυτόν δεν θα ήμουν σε θέση να ολοκληρώσω την παρούσα διπλωματική εργασία.
